\documentclass{article}
\usepackage[hungarian]{babel}
\usepackage{t1enc}
\usepackage{amsthm}
\usepackage{algpseudocode}
\usepackage{algorithm}
\usepackage{listings}
\usepackage{lipsum}
\usepackage{float}

\theoremstyle{plain}
\newtheorem{tetel}{Tétel} %+
\newtheorem{lemm}[tetel]{Lemma} %+

\theoremstyle{definition}
\newtheorem{defe}{Definíció} %+

\theoremstyle{remark}
\newtheorem{fel}{Feladat}[section] %+

\newtheorem*{megj}{Megjegyzés}

\newfloat{forraskod}{hbt}{lop}
\floatname{forraskod}{Forráskód}
\floatname{algorithm}{Algoritmus}
\renewcommand{\listalgorithmname}{Algoritmusok listája}

\begin{document}
%%%%%%%%%%%%%%%%%
%------------------%
\listof{forraskod}{Verbatimok listája}
%------------------%
\listofalgorithms
%------------------%
\section{Első rész}
\begin{fel}

\end{fel}
%
\begin{fel}

\end{fel}
%
\begin{fel}

\end{fel}
%%%%%%%%%%%
\section{Második rész}
\begin{tetel}
\hphantom\\\\
Ez a tétel első sora. \\
Ez a tétel második sora. \verb|| \\
Lorem ipsum.
\end{tetel}
%-
\begin{proof}

\end{proof}
%-
\begin{defe}

\end{defe}
%-
\begin{tetel}
\hphantom\\\\
Ez a tétel első sora. \verb|| \\ 
Ez a tétel második sora. \\
Lorem ipsum.
\end{tetel}
%-
\begin{defe}

\end{defe}
%-
\begin{lemm}

\end{lemm}
%-
\begin{fel}

\end{fel}
%-
\begin{fel}

\end{fel}
%-
\begin{fel}

\end{fel}
%------------------%
\begin{megj}

\end{megj}
%------------------%
\clearpage
\begin{forraskod}
\caption{Első verbatim}
\begin{verbatim}
\begin{tetel}
Ez a tétel első sora.
Ez a tétel második sora.
Lorem ipsum.
\end{tetel}
\end{verbatim}
%-
\end{forraskod}
\begin{forraskod}
\caption{Második verbatim}
\begin{verbatim}
\begin{enumerate}
\item Első
\item Item második
\end{enumerate}
\end{verbatim}
\end{forraskod}
\lipsum[1-3]
\newpage
%------------------%
\begin{forraskod}
\caption{Python-kód}
\lstinputlisting[language=python, tabsize=4, numbers=left, stepnumber=3, frame=shadowbox]{forrasfajlok/binsearch.py}
\end{forraskod}
\clearpage
%------------------%
\begin{algorithm}
\caption{Gyorsrendezés}
\begin{algorithmic}[2]
\Procedure{QUICKSORT}{@A,a,b}
\Require A írható tömb
\Require 1 $\leq$  a $\leq$ b $\leq$ Hossz[A] indexek
\Ensure a-b indextartományt rendezzük
\If{a=b}
\State\Return A \Comment egyelemű tömb mindig rendezett
\Else
\State\Call{FELOSZT}{@A,a,b,A(a),@q} \Comment k-A(a), a tartomány első eleme
\State\Call{QUICKSORT}{@A,a,q}
\State\Call{QUICKSORT}{@A,q+1,b}
\State\Return A
\EndIf
\EndProcedure
\end{algorithmic}
\end{algorithm}
%-------------------%
\begin{algorithm}
\begin{algorithmic}[2]
\Procedure{FELOSZT}{@A,a,b,k,@q}
\Require A írható tömb
\Require 1 $\leq$  a $\leq$ b $\leq$ Hossz[A] indexek
\Require k A-beli kulcs
\Ensure A átrendezése és q választása úgy, hogy: a - q indextartomány elemei $\leq$ k, (q+1) - b indextartomány elemei $\geq$ k
\State i $\leftarrow$ a-1
\State j $\leftarrow$ b+1
\While{i<j} \Comment ekvivalens: while true do...
\Repeat
\State INC(i)
\Until{A(i) $\geq$ k}
\Repeat
\State DEC(j)
\Until{A(j) $\leq$ k}
\If{i < j}
\State A(i) $\Leftrightarrow$ A(j) csere
\Else
\State q $\leftarrow$ j
\State \Return (A,q)
\EndIf
\EndWhile
\EndProcedure
\caption{Felosztás}
\end{algorithmic}
\end{algorithm}
\clearpage
%------------------%
\algblockdefx[owndowhile]{Do}{While}{\textbf{do}}[1]{\textbf{while}~#1~}

\begin{algorithmic}
\Do 
\State INC(i)
\While{a $\ne$ i}
\end{algorithmic}
%------------------%
\vspace{0.5cm}
\algblockdefx[ownswitch]{Switch}{EndSwitch}[1]{\textbf{switch}~#1~}{}

\algcblockdefx[ownswitch]{ownswitch}{Case}{EndSwitch}[2]{~~~~\textbf{case} #1\textbf{:}~#2~}{}

\begin{algorithmic}
\Switch{a = b}
\Case {true}{INC(a)}
\Case {false}{DEC(b)}
\EndSwitch

\end{algorithmic}
%%%%%%%%%%%%%%%%%
\end{document}

