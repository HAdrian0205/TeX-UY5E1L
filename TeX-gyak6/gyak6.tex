\documentclass{article}
\usepackage[magyar]{babel}
\usepackage{t1enc}
\frenchspacing
\usepackage{amsmath,stackengine}
\stackMath
\usepackage{amsfonts, amssymb}
\usepackage{mathtools}
\usepackage{enumitem}
\usepackage{xcolor}


\begin{document}
\section{Bevezető}
\begin{enumerate}[label=\alph*)]
\item Az $\dfrac{1}{n^2}$ sorösszege:
\begin{equation}
\sum_{n=1}^{\infty} = \dfrac{\pi^2}{6}.
\end{equation}
\item Az $n!$ ($n$ faktoriális) a számok szorzata 1-től $n$-ig, azaz
\begin{equation}
n! := \prod_{k=1}^n k = 1\cdot 2 \cdot ... \cdot n.
\end{equation}
Konvenció szerint $0! = 1.$
\item Legyen $0 \leq k \leq n$. A binomiális együttható
\begin{equation}
\begin{pmatrix}
n \\
k
\end{pmatrix}
:= \dfrac{n!}{k! \cdot (n-k)!}, 
\end{equation}
ahol a faktoriálist (\textcolor{red}{1}) szerint definiáljuk.

\item Az előjel- azaz szignum függvényt a következőképpen definiáljuk:
\begin{equation}
sgn(x) := 
\begin{cases}
1, & \text{ha x > 0,} \\
0, & \text{ha x = 0,} \\
-1, & \text{ha x < 0.}
\end{cases}
\end{equation}
\end{enumerate}
\newpage
\section{Determináns}
\begin{enumerate}[label=\alph*)] 
\item Legyen
\begin{equation}
[n]:= \{ 1,2,\cdots, n \}
\end{equation}
a természetes számok halmaza 1-től $n$-ig.
\item Egy n-edrendű \textit{permutáció} $\sigma$ egy bijekció $[n]$-ből $[n]$-be. Az $n$-edrendű permutációk halmazát, az ún. szimmetrikus csoportot, $S_n$-nel jelöljük.
\item Egy $\sigma \in S_n$ permutációban inverziónak nevezünk egy $(i,j)$ párt, ha $i < j$ de $\sigma_i > \sigma_j$.
\item Egy $\sigma \in S_n$ permutáció paritásának az inverziók számát nevezzük:
\begin{equation}
I(\sigma) := 
\begin{vmatrix}
\{(i,j)\text{ | }i,j \in [n], i<j, \sigma i> \sigma j\} 
\end{vmatrix}
\end{equation}
\item Legyen $A \in \mathbb{R}^{n\times n}$, egy $n\times n$-es (négyzetes) valós mátrix:
\begin{equation}
A=
\begin{pmatrix}
a_{11} & a_{12} & \cdots & a_{1n} \\
a_{21} & a_{22} & \cdots & a_{2n} \\
\vdots & \vdots & \ddots & \vdots \\
a_{n1} & a_{n2} & \cdots & a_{nn}
\end{pmatrix}
\end{equation}
Az $A$ mátrix determinánsát a következőképpen definiáljuk:
\begin{equation}
det(A) = 
\begin{vmatrix}
a_{11} & a_{12} & \cdots & a_{1n} \\
a_{21} & a_{22} & \cdots & a_{2n} \\
\vdots & \vdots & \ddots & \vdots \\
a_{n1} & a_{n2} & \cdots & a_{nn}
\end{vmatrix} := \sum_{\sigma \in S_n} (-1)^{I(\sigma)} \prod_{i=1}^n a_{i\sigma_i}
\end{equation}
\end{enumerate}
\newpage
\section{Logikai azonosság}
Tekintsük az $L = \{0,1\}$ halmazt, és rajta a következő, igazságtáblával definiált műveleteket:
\begin{equation}
\begin{tabular}{c||c}
x & $\bar{x}$ \\ \hline
0 & 1 \\
1 & 0 \\
\end{tabular} \hspace{0.5cm}
\begin{tabular}{cc||c|c|c}
x & y & x $\vee$ y & x $\wedge$ y & x $\rightarrow$ y \\ \hline
0 & 0 & 0 & 0 & 1 \\
0 & 1 & 1 & 0 & 1 \\
1 & 0 & 1 & 0 & 0 \\
1 & 1 & 1 & 1 & 1
\end{tabular}
\end{equation}
Legyenek $a,b,c,d \in L$. Belátjuk a következő azonosságot:
\begin{equation}
(a \wedge b \wedge c) \rightarrow d = a \rightarrow (b \rightarrow (c \rightarrow d)).
\end{equation}
A következő azonosságokat bizonyítás nélkül használjuk:
\begin{equation}
x \rightarrow y = \bar{x} \vee y
\end{equation}
\begin{equation}
\overline{x \vee y} = \bar{x} \wedge \bar{y} \hspace{0.5cm}
\overline{x \wedge y} = \bar{x} \vee \bar{y}
\end{equation}
A (\textcolor{red}{10}) bal oldala, (\textcolor{red}{11}) felhasználásával
\begin{equation}
(a \wedge b \wedge c) \rightarrow d \underset{(\textcolor{red}{11})}{=} \overline{a \wedge b \wedge c} \vee d \underset{(\textcolor{red}{12})}{=} (\bar{a} \vee \bar{b} \vee \bar{c}) \vee d.
\end{equation}
A (\textcolor{red}{10}) jobb oldala, (\textcolor{red}{11}) ismételt felhasználásával
\begin{equation}
\begin{aligned}
a \rightarrow (b \rightarrow (c \rightarrow d)) &= \bar{a} \vee (b \rightarrow  (c \rightarrow d))\\ 
&= \text{	} \bar{a} \vee (\bar{b} \vee (c \rightarrow d))\\
&= \text{	} \bar{a} \vee (\bar{b} \vee (\bar{c} \vee v)),
\end{aligned}
\end{equation}
ami a $\vee$ asszociativitása miatt egyenlő (\textcolor{red}{13}) egyenlettel.

\section{Binomiális tétel}
\begin{subequations}
\begin{align}
(a+b)^{n+1} &= (a+b) \cdot 
\left(
\sum_{k=0}^{n}
\begin{pmatrix}
n \\
k
\end{pmatrix}
a^{n-k}b^k
\right) \\
&= \cdots \nonumber\\
&= \sum_{k=0}^n
\begin{pmatrix}
n \\
k
\end{pmatrix}
a^{(n+1)-k}b^{k} + \sum_{k=1}^{n+1}
\begin{pmatrix}
n \\
k-1
\end{pmatrix}
a^{(n+1)-k}b^{k} \\
&= \cdots \nonumber\\
&= 
\begin{pmatrix}
n+1 \\
0
\end{pmatrix}
a^{n+1-0}b^0 + \sum_{k=1}^{n}
\begin{pmatrix}
n+1 \\
k
\end{pmatrix}
a^{(n+1)-k}b^{k} \\ \nonumber
&+
\begin{pmatrix}
n+1 \\
n+1
\end{pmatrix}
a^{n+1-(n+1)}b^{n+1}. \\
&= \sum_{k=0}^{n+1}
\begin{pmatrix}
n+1 \\
k
\end{pmatrix}
a^{(n+1)-k}b^k.
\end{align} 
\end{subequations}
\end{document}