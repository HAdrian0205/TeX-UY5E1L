\documentclass{article}
\usepackage[english, magyar]{babel}
\usepackage{t1enc}
\usepackage{xcolor}
\usepackage{graphicx}
\usepackage{lipsum}
\usepackage{hulipsum}
\usepackage{blindtext}
\title{I. Gyakorlat}

\begin{document}
\maketitle
\selectlanguage{magyar}
\frenchspacing
\texttt{\textbf{Ez az első szöveg.}}
\\
\textsc{\textit{Ez a második szöveg.}}
\\
\textsl{\textsf{Ez a harmadik szöveg.}}
\\
\emph{A \textit{fiú} \textbf{elment} a \textsc{boltba}.}
\\
\reflectbox{Ez a negyedik szöveg.}
\\
\reflectbox{\framebox{Ez a negyedik szöveg bekeretezve.}}
\\
\scalebox{1}[-1]{\scalebox{2.0}{Ez az ötödik szöveg.}}
\\
Ebben a szövegben egy szó \rotatebox{90}{elfordul}, és egy másik is \rotatebox{270}{elfordul}.
\\
\selectlanguage{english}{\colorbox{black}{\color{white}{\parbox{12cm}{"\blindtext[1]"}}}}
\\
\fcolorbox{red}{white}{\color{red}{Ez a hetedik szöveg, ami piros.}}
\\
\begin{flushright}
\today{"\hulipsum[1]"}
\end{flushright}
\today{\selectlanguage{latin}{\linespread{2}{\lipsum[1]}}}
\\

\end{document}

